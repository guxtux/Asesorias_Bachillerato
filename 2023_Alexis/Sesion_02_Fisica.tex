\documentclass[12pt]{beamer}
\usepackage{Estilos/BeamerFC}
\usepackage{Estilos/ColoresLatex}
\usetheme{Warsaw}
\usecolortheme{seahorse}
%\useoutertheme{default}
\setbeamercovered{invisible}
% or whatever (possibly just delete it)
\setbeamertemplate{section in toc}[sections numbered]
\setbeamertemplate{subsection in toc}[subsections numbered]
\setbeamertemplate{subsection in toc}{\leavevmode\leftskip=3.2em\rlap{\hskip-2em\inserttocsectionnumber.\inserttocsubsectionnumber}\inserttocsubsection\par}
\setbeamercolor{section in toc}{fg=blue}
\setbeamercolor{subsection in toc}{fg=blue}
\setbeamercolor{frametitle}{fg=blue}
\setbeamertemplate{caption}[numbered]

\setbeamertemplate{footline}
\beamertemplatenavigationsymbolsempty
\setbeamertemplate{headline}{}


\makeatletter
\setbeamercolor{section in foot}{bg=gray!30, fg=black!90!orange}
\setbeamercolor{subsection in foot}{bg=blue!30}
\setbeamercolor{date in foot}{bg=black}
\setbeamertemplate{footline}
{
  \leavevmode%
  \hbox{%
  \begin{beamercolorbox}[wd=.333333\paperwidth,ht=2.25ex,dp=1ex,center]{section in foot}%
    \usebeamerfont{section in foot} \insertsection
  \end{beamercolorbox}%
  \begin{beamercolorbox}[wd=.333333\paperwidth,ht=2.25ex,dp=1ex,center]{subsection in foot}%
    \usebeamerfont{subsection in foot}  \insertsubsection
  \end{beamercolorbox}%
  \begin{beamercolorbox}[wd=.333333\paperwidth,ht=2.25ex,dp=1ex,right]{date in head/foot}%
    \usebeamerfont{date in head/foot} {T1 - Segunda presentación} \hspace*{2em}
    \insertframenumber{} / \inserttotalframenumber \hspace*{2ex} 
  \end{beamercolorbox}}%
  \vskip0pt%
}
\makeatother

\makeatletter
\patchcmd{\beamer@sectionintoc}{\vskip1.5em}{\vskip0.8em}{}{}
\makeatother
\usepackage{pifont}
\newcommand{\cmark}{\ding{51}}%
\newcommand{\xmark}{\ding{55}}%

\makeatletter
\setbeamertemplate{footline}
{
  \leavevmode%
  \hbox{%
  \begin{beamercolorbox}[wd=.333333\paperwidth,ht=2.25ex,dp=1ex,center]{section in foot}%
    \usebeamerfont{section in foot} \insertsection
  \end{beamercolorbox}%
  \begin{beamercolorbox}[wd=.333333\paperwidth,ht=2.25ex,dp=1ex,center]{subsection in foot}%
    \usebeamerfont{subsection in foot}  \insertsubsection
  \end{beamercolorbox}%
  \begin{beamercolorbox}[wd=.333333\paperwidth,ht=2.25ex,dp=1ex,right]{date in head/foot}%
    \usebeamerfont{date in head/foot} \insertshortdate{} \hspace*{2em}
    \insertframenumber{} / \inserttotalframenumber \hspace*{2ex} 
  \end{beamercolorbox}}%
  \vskip0pt%
}
\makeatother

\setbeamertemplate{navigation symbols}{}
\date{14 de marzo}

\title{Sesión 2. Física}
\subtitle{Asesoría}

\begin{document}

\maketitle
\fontsize{14}{14}\selectfont
\spanishdecimal{.}

\section*{Contenido}
\frame[allowframebreaks]{\tableofcontents[currentsection, hideallsubsections]}

\section{Mediciones y tablas}
\frame{\tableofcontents[currentsection, hideothersubsections]}
\subsection{Ejercicio 1}

\begin{frame}
\frametitle{Enunciado Problema 1}
En una competencia olímpica de salto largo seleccionaron a $7$ jueces para medir un salto cuyas mediciones se muestran en el siguiente cuadro.
\end{frame}
\begin{frame}
\frametitle{Enunciado del Problema 1}
Obtener el:
\pause
\setbeamercolor{item projected}{bg=black,fg=white}
\setbeamertemplate{enumerate items}{%
\usebeamercolor[bg]{item projected}%
\raisebox{1.5pt}{\colorbox{bg}{\color{fg}\footnotesize\insertenumlabel}}%
}
\begin{enumerate}[<+->]
\item Valor promedio.
\item Error absoluto.
\item Error relativo.
\item Error porcentual.
\item Desviación media.
\end{enumerate}
\end{frame}
\begin{frame}
\frametitle{Datos del salto}
\begin{table}
\renewcommand{\arraystretch}{1}
\centering
\begin{tabular}{| c |} \hline
Mediciones en $\unit{\meter}$ \\ \hline
$\SI{8.90}{\meter}$ \\ \hline
$\SI{8.92}{\meter}$ \\ \hline
$\SI{8.88}{\meter}$ \\ \hline
$\SI{8.79}{\meter}$ \\ \hline
$\SI{8.91}{\meter}$ \\ \hline
$\SI{8.93}{\meter}$ \\ \hline
$\SI{8.89}{\meter}$ \\ \hline
\end{tabular}
\end{table}
\end{frame}
\begin{frame}
\frametitle{Valor promedio}
Se define el valor promedio $\bar{x}$ de un conjunto de datos, como el cociente de la suma de los datos dividido entre el total de datos, es decir:
\pause
\begin{align*}
\bar{x} = \nsum_{i=1}^{n} \dfrac{x_{i}}{n}
\end{align*}
\end{frame}
\begin{frame}
\frametitle{Error absoluto}
Para obtener el valor absoluto de una medición, se requiere comparar la diferencia de una magnitud que consideraremos como la \enquote{exacta} y cada una de las mediciones que se tienen:
\pause
\begin{align*}
\text{error absoluto} = \abs{\text{valor}_{\text{exacto}} - x_{i}}
\end{align*}
\end{frame}
\begin{frame}
\frametitle{Del error absoluto}
El error absoluto, nos devuelve un valor de la diferencia, pero como tal, no sabemos qué tanta es esa diferencia.
\\
\bigskip
\pause
Para ello se utiliza el error relativo o el error porcentual.
\end{frame}
\begin{frame}
\frametitle{Error relativo}
Para obtener el valor relativo de una medición, tomamos el error absoluto y se divide entre el valor que se considera como \enquote{exacto}:
\pause
\begin{align*}
\text{error relativo} = \dfrac{\abs{\text{valor}_{\text{exacto}} - x_{i}}}{\text{valor}_{\text{exacto}}}
\end{align*}
\end{frame}
\begin{frame}
\frametitle{Error porcentual}
Es el valor del error relativo expresado en $\%$, para ello, se multiplica por $100\%$:
\pause
\begin{align*}
\text{error porcentual} = \dfrac{\abs{\text{valor}_{\text{exacto}} - x_{i}}}{\text{valor}_{\text{exacto}}} \times 100 \%
\end{align*}
\end{frame}
\begin{frame}
\frametitle{Desviación media}
Consideramos la desviación media como el valor absoluto de la diferencia entre el promedio de un conjunto de datos y cada uno de los datos.
\pause
\begin{align*}
\text{desviación media} = \abs{\bar{x} - x_{i}}
\end{align*}
\end{frame}

\begin{frame}
\frametitle{Solución al problema 1}
Como el enunciado no indica qué valor se debe de considerar como el valor \enquote{exacto}, tenemos la libertad de elegir ese valor, a partir de nuestro criterio:
\pause
\setbeamercolor{item projected}{bg=cerise,fg=white}
\setbeamertemplate{enumerate items}{%
\usebeamercolor[bg]{item projected}%
\raisebox{1.5pt}{\colorbox{bg}{\color{fg}\footnotesize\insertenumlabel}}%
}
\begin{enumerate}[<+->]
\item El valor más bajo.
\item El valor más alto.
\item El valor promedio.
\end{enumerate}
\end{frame}
\begin{frame}
\frametitle{Calculando el valor promedio}
Ocupando la expresión que indicamos para el valor promedio $\bar{x}$, tendremos que:
\pause
\begin{eqnarray*}
\begin{aligned}
\bar{x} &= \nsum_{i=1}^{7} \dfrac{x_{i}}{7} = \pause \dfrac{\SI{8.90}{\meter} {+} \SI{8.92}{\meter} {+} \SI{8.88}{\meter} {+} \ldots + \SI{8.89}{\meter}}{7} = \\[0.5em] \pause
&= \dfrac{\SI{62.22}{\meter}}{7} = \\[0.5em] \pause
&= \SI{8.88}{\meter}
\end{aligned}
\end{eqnarray*}
\pause
Usaremos este valor como el \enquote{exacto} para calcular los errores.
\end{frame}
\begin{frame}
\frametitle{Error absoluto}
\begin{align*}
\text{error absoluto} = \abs{\SI{8.88}{\meter} - x_{i}}
\end{align*}
\pause
\begin{table}
\renewcommand{\arraystretch}{0.8}
\centering
\begin{tabular}{| c | c |} \hline
Mediciones en $\unit{\meter}$ & Error absoluto en $\unit{\meter}$\\ \hline
$8.90$ & $0.02$ \\ \hline
$8.92$ & $0.04$ \\ \hline
$8.88$ & $0$ \\ \hline
$8.79$ & $0.09$ \\ \hline
$8.91$ & $0.03$ \\ \hline
$8.93$ & $0.05$ \\ \hline
$8.89$ & $0.01$ \\ \hline
\end{tabular}
\end{table}    
\end{frame}
\begin{frame}
\frametitle{Error relativo}
\begin{align*}
\text{error relativo} = \dfrac{\abs{\SI{8.88}{\meter} - x_{i}}}{\SI{8.88}{\meter}}
\end{align*}
\pause
\vspace*{-1cm}
\begin{table}
\renewcommand{\arraystretch}{0.8}
\centering
\begin{tabular}{| c | c | c |} \hline
Mediciones en $\unit{\meter}$ & Error absoluto en $\unit{\meter}$& Error relativo\\ \hline
$8.90$ & $0.02$ & $0.00225$ \\ \hline
$8.92$ & $0.04$ & $0.00450$\\ \hline
$8.88$ & $0$ & $0$ \\ \hline
$8.79$ & $0.09$ & $0.01013$ \\ \hline
$8.91$ & $0.03$ & $0.00337$ \\ \hline
$8.93$ & $0.05$ & $0.00563$ \\ \hline
$8.89$ & $0.01$ & $0.00112$ \\ \hline
\end{tabular}
\end{table}    
\end{frame}
\begin{frame}
\frametitle{Error porcentual}
\begin{align*}
\text{error porcentual} = \dfrac{\abs{\SI{8.88}{\meter} - x_{i}}}{\SI{8.88}{\meter}} \times 100 \%
\end{align*}
\end{frame}
\begin{frame}
\frametitle{Error porcentual}
\begin{table}
\renewcommand{\arraystretch}{0.8}
\centering
\begin{tabular}{| p{2cm} | p{2cm} | p{2cm} | p{2cm} |} \hline
\multicolumn{1}{|p{2cm}|}{\centering Mediciones \\ en $\unit{\meter}$} & \multicolumn{1}{|p{2cm}|}{\centering Error absoluto \\ en $\unit{\meter}$} & \multicolumn{1}{|p{2cm}|}{\centering Error \\ relativo} & \multicolumn{1}{|p{2cm}|}{\centering Error \\ porcentual} \\ \hline
$8.90$ & $0.02$ & $0.00225$ & $0.225 \%$ \\ \hline
$8.92$ & $0.04$ & $0.00450$ & $0.455 \%$ \\ \hline
$8.88$ & $0$ & $0$ & $0.0 \%$  \\ \hline
$8.79$ & $0.09$ & $0.01013$ & $1.013 \%$  \\ \hline
$8.91$ & $0.03$ & $0.00337$ & $0.337 \%$  \\ \hline
$8.93$ & $0.05$ & $0.00563$ & $0.563 \%$  \\ \hline
$8.89$ & $0.01$ & $0.00112$ & $0.112 \%$  \\ \hline
\end{tabular}
\end{table}    
\end{frame}
\begin{frame}
\frametitle{Desviación media}
\begin{align*}
\text{desviación media} = \abs{\SI{8.88}{\meter} - x_{i}}
\end{align*}    
\end{frame}
\begin{frame}
\frametitle{Desviación media}
\begin{table}
\renewcommand{\arraystretch}{0.8}
\centering
\begin{tabular}{| p{2cm} | p{2cm} | } \hline
\multicolumn{1}{|p{2cm}|}{\centering Mediciones \\ en $\unit{\meter}$} & \multicolumn{1}{|p{2cm}|}{\centering Desviación \\ media en $\unit{\meter}$} \\ \hline
$8.90$ & $0.02$ \\ \hline
$8.92$ & $0.04$ \\ \hline
$8.88$ & $0$ \\ \hline
$8.79$ & $0.09$ \\ \hline
$8.91$ & $0.03$ \\ \hline
$8.93$ & $0.05$ \\ \hline
$8.89$ & $0.01$ \\ \hline
\end{tabular}
\end{table}    
\end{frame}    
\begin{frame}
\frametitle{Caso de la desviación media}
En estadística se acostumbra reportar la desviación media del conjunto de datos, como medida de dispersión.
\\
\bigskip
\pause
Se utiliza la expresión:
\begin{align*}
\text{desviación media} = \dfrac{\displaystyle \nsum_{i=1}^{n} \abs{\bar{x} - x_{i}}}{n} 
\end{align*}
\end{frame}
\begin{frame}
\frametitle{Reportando la desviación media}
En el caso del ejercicio 1, se tiene que:
\pause
\begin{eqnarray*}
\begin{aligned}
\text{desviación media} &= \dfrac{\displaystyle \nsum_{i=1}^{7} \abs{\SI{8.88}{\meter} - x_{i}}}{n} = \\[0.5em] \pause
&= \dfrac{\SI{0.02}{\meter} {+} \SI{0.04}{\meter} {+} \ldots {+} \SI{0.01}{\meter}}{7} = \\[0.5em] \pause
&= \dfrac{\SI{0.24}{\meter}}{7} = \\[0.5em] \pause
&= \SI{0.0342}{\meter}
\end{aligned}
\end{eqnarray*}
\end{frame}


\section{Vectores}
\frame{\tableofcontents[currentsection, hideothersubsections]}
\subsection{Definición}

\begin{frame}
\frametitle{¿Qué es un vector?}
Los vectores se definen como expresiones matemáticas que poseen magnitud, dirección y sentido.
\\
\bigskip
\pause
Los vectores se representan por flechas en las ilustraciones, normalmente se distinguen de las cantidades escalares mediante el uso de negritas ($\vb{P}$).
\end{frame}
\begin{frame}
\frametitle{Varios vectores}
\begin{figure}
    \centering
    \begin{tikzpicture}
        \draw [-stealth, thick] (0, 0) -- (3, 2) node [midway, above] {$\va{P}$};
        \draw (0, 0) -- (2, 0);
        \draw (0.5, 0) arc(0:30:0.6);
        \node at (0.8, 0.2) {$\theta$};

        \draw [stealth-, thick] (7, 1) -- (9, 1) node [midway, above] {$\va{A}$};
    \end{tikzpicture}
\end{figure}
\end{frame}
\begin{frame}
\frametitle{Escribiendo vectores}
En la escritura a mano, un vector puede caracterizarse dibujando una pequeña flecha arriba de la letra usada para representarlo ($\va{P}$).
\\
\bigskip
\pause
La magnitud de un vector determina la longitud de la flecha correspondiente.
\end{frame}
\begin{frame}
\frametitle{Definiendo un vector}
\begin{figure}
    \centering
    \begin{tikzpicture}
        \draw [-stealth, thick] (0, 0) -- (3, 3) node [midway, above] {$\va{P}$};
        \draw [dashed] (-1, -1) -- (4, 4); 
        \draw (0, 0) -- (2, 0);
        \draw (0.5, 0) arc(0:30:0.8);
        \node at (0.8, 0.2) {$\theta$};
        \node at (5, 4) {Línea de acción};
    \end{tikzpicture}
\end{figure}
\end{frame}
\begin{frame}
\frametitle{Dirección de un vector}
La dirección de un vector representa el ángulo de inclinación con respecto a una línea horizontal medidad en el sentido antihorario.
\pause
\begin{figure}
    \centering
    \begin{tikzpicture}
        \draw [-stealth, thick] (0, 0) -- (2, 2) node [midway, above] {$\va{A}$};
        \draw [dashed] (-0.5, -0.5) -- (2.5, 2.5); 
        \draw (0, 0) -- (2, 0);
        \draw (0.5, 0) arc(0:45:0.5);
        \node at (1.5, 0.2) {$\theta = \ang{45}$};
        
        \pause
        
        \draw [-stealth, thick] (6, 0) -- (4, 2) node [midway, above] {$\va{B}$};
        \draw [dashed] (7, -1) -- (3, 3); 
        \draw (6, 0) -- (8.5, 0);
        \draw (6.5, 0) arc(0:135:0.55);
        \node at (7.5, 0.2) {$\theta = \ang{135}$};
    \end{tikzpicture}
\end{figure}
\end{frame}

\subsection{Tipos de vectores}

\begin{frame}
\frametitle{Vectores colineales}
Son aquellos que están contenidos en una misma línea de acción.
\pause
\begin{figure}
    \centering
    \begin{tikzpicture}
        \draw [dashed] (0, 0) -- (10, 0);
        \draw [-stealth, thick] (1, 0) -- (2, 0) node [above, midway] {$\va{A}$};
        \draw [stealth-, thick] (4, 0) -- (5, 0) node [above, midway] {$\va{B}$};
        \draw [-stealth, thick] (7, 0) -- (8, 0) node [above, midway] {$\va{C}$};
    \end{tikzpicture}
\end{figure}
\end{frame}
\begin{frame}
\frametitle{Vectores concurrentes}
Son aquellos cuyas líneas de acción se cortan en un mismo punto.
\pause
\begin{figure}
    \centering
    \begin{tikzpicture}
        \draw [dashed] (0, 0) -- (6, 0);
        \draw [-stealth, thick] (1, 0) -- (2, 0) node [above, midway] {$\va{A}$};
        \draw [stealth-, thick] (5, 0) -- (6, 0) node [above, midway] {$\va{B}$};
        
        \draw [dashed] (3, 3) -- (3, -2);
        \draw [-stealth, thick] (3, 3) -- (3, 2) node [left, midway] {$\va{C}$};
    \end{tikzpicture}
\end{figure}
\end{frame}
\begin{frame}
\frametitle{Vectores coplanares}
Son aquellos que se encuentran en un mismo plano:
\pause
\begin{figure}
    \centering
    \begin{tikzpicture}
        \draw[fill=aquamarine] (0, 0) rectangle (5, 3);
        \draw [-stealth, thick] (0.5, 1) -- (1.5, 1) node [above, midway] {$\va{a}$};
        \draw [-stealth, thick] (4, 2) -- (3, 1) node [above, midway] {$\va{b}$};
        \draw [-stealth, thick] (2.5, 1) -- (2.5, 2) node [left, midway] {$\va{c}$};
    \end{tikzpicture}
\end{figure}
\end{frame}

\end{document}