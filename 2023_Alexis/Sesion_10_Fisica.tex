\documentclass[12pt]{beamer}
\usepackage{Estilos/BeamerFC}
\usepackage{Estilos/ColoresLatex}
\input{Preambulos/preambulo_Beamer_Warsaw_seahorse}
\usepackage{pifont}
\newcommand{\cmark}{\ding{51}}%
\newcommand{\xmark}{\ding{55}}%

\makeatletter
\setbeamertemplate{footline}
{
  \leavevmode%
  \hbox{%
  \begin{beamercolorbox}[wd=.333333\paperwidth,ht=2.25ex,dp=1ex,center]{section in foot}%
    \usebeamerfont{section in foot} \insertsection
  \end{beamercolorbox}%
  \begin{beamercolorbox}[wd=.333333\paperwidth,ht=2.25ex,dp=1ex,center]{subsection in foot}%
    \usebeamerfont{subsection in foot}  \insertsubsection
  \end{beamercolorbox}%
  \begin{beamercolorbox}[wd=.333333\paperwidth,ht=2.25ex,dp=1ex,right]{date in head/foot}%
    \usebeamerfont{date in head/foot} \insertshortdate{} \hspace*{2em}
    \insertframenumber{} / \inserttotalframenumber \hspace*{2ex} 
  \end{beamercolorbox}}%
  \vskip0pt%
}
\makeatother

\sisetup{
  per-mode=fraction,
  fraction-function=\tfrac
}

\setbeamertemplate{navigation symbols}{}
\date{27 de abril}

% \sisetup{math-rm=\symup,detect-all}
\sisetup{detect-all, math-rm = \ensuremath}

\title{Sesión 10. Física}
\subtitle{Asesoría}

\begin{document}

\maketitle
\fontsize{14}{14}\selectfont
\spanishdecimal{.}

\section*{Contenido}
\frame[allowframebreaks]{\tableofcontents[currentsection, hideallsubsections]}

\section{Primera condición de equilibrio}
\frame{\tableofcontents[currentsection, hideothersubsections]}
\subsection{Ejercio 8 - Guía}

\begin{frame}
\frametitle{Enunciado del Ejercicio 8}
Un anuncio de \SI{200}{\newton} se cuelga de un techo como se muestra en la figura, de manera que las cuerdas que lo sostienen forman un ángulo de \ang{90}.
\\
\bigskip
\pause
¿Cuál es la tensión en cada segmnento de la cuerda?
\end{frame}
\begin{frame}
\frametitle{Figura del Ejercicio 8}
\begin{figure}
  \centering
  \includegraphics[scale=0.9]{Imagenes/DCL_Problema_08.png}
\end{figure}
\end{frame}
\begin{frame}
\frametitle{Construyedo el DCL}
Para el DCL debemos de considerar las fuerzas involucradas en el ejercicio, tomemos en cuenta que: \pause el enunciado nos indica el peso del anuncio:
\begin{align*}
W = \SI{200}{\newton}
\end{align*}
\end{frame}
\begin{frame}
\frametitle{Construyedo el DCL}
Así mismo, sabemos cuánto vale el ángulo que se forma entre la cuerda $1$ y la parte superior del anuncio:
\pause
\begin{eqnarray*}
\begin{aligned}
\theta_{1} &+ \ang{90} + \theta_{2} = \ang{180} \\[0.5em] \pause
\theta_{1} &= \ang{180} - \ang{90} - \theta_{2} = \\[0.5em] \pause
\theta_{1} &= \ang{180} - \ang{90} - \ang{50} = \\[0.5em] \pause
\theta_{1} &= \ang{40}
\end{aligned}
\end{eqnarray*} 
\end{frame}
\begin{frame}
\frametitle{Diagrama de cuerpo libre}
\begin{figure}
\centering
\begin{tikzpicture}[scale=1.3]
  \draw (-1, 0) -- (1.5, 0);
  \draw [-stealth, thick, color=ao](0, 0) -- (0, -2) node [right, midway] {\small{$-W$}};
  
  \pause
  \draw [thick, -stealth, color=red] (0, 0) -- (0.986, 1.175) node [above=0.1, midway] {\small{$T_{2}$}};
  \draw (-1.4, 1.175) -- (1.2, 1.175);
  % \draw (0.986, 1.175) -- (-0.985, 0.5);
  \draw [color=red] (0.5, 0) arc(0:50:0.5);
  \node at (0.7, 0.2) [color=red] {\small{$\theta_{2}$}};

  \pause
  \draw [thick, -stealth, color=red] (0, 0) -- (-0.985, 1.175) node [above, midway] {\small{$T_{1}$}};
  
  \draw [color=red] (-0.4, 0) arc(180:130:0.4);
  \node at (-0.6, 0.2) [color=red] {\small{$\theta_{1}$}};

\end{tikzpicture}
\end{figure}
\end{frame}
\begin{frame}
\frametitle{Listando las fuerzas}
Presentamos la tabla con las fuerzas involucradas y los ángulos que forman:
\pause
\begin{table}
\centering
\begin{tabular}{c | c}
Fuerza & Ángulo \\ \hline
$T_{1}$ & \ang{40} \\ \hline
$T_{2}$ & \ang{50} \\ \hline
$W$ & \ang{270} \\ \hline
\end{tabular}
\end{table}
\end{frame}
\begin{frame}
\frametitle{Descomponiendo los vectores}
Ahora procedemos a hacer la descomposición de los vectores tanto en la dirección del eje $x$ como del eje $y$.
\end{frame}
\begin{frame}
\frametitle{Descomponiendo los vectores}
\begin{table}
\centering
\begin{tabular}{c | c | c}
Componente & Expresión & Sustitución \\ \hline
$T_{1x}$ & $(- \cos \ang{40})(T_{1})$ & $(-0.766) (T_{1})$ \\ \hline
$T_{1y}$ & $(\sin \ang{40})(T_{1})$ & $(0.642) (T_{1})$ \\ \hline
$T_{2x}$ & $(\cos \ang{50})(T_{2})$ & $(0.642) (T_{2})$ \\ \hline
$T_{2y}$ & $(\sin \ang{50})(T_{2})$ & $(0.766) (T_{2})$ \\ \hline
\end{tabular}
\end{table}
\end{frame}
\begin{frame}
\frametitle{Descomponiendo los vectores}
\begin{table}
\centering
\begin{tabular}{c | c | c}
Componente & Expresión & Sustitución \\ \hline
$W_{x}$ & $(\cos \ang{270})(W)$ & $(0) (W)$ \\ \hline
$W_{y}$ & $(\sin \ang{270})(W)$ & $(-1) (\SI{200}{\newton})$ \\ \hline
\end{tabular}
\end{table}
\end{frame}
\begin{frame}
\frametitle{Condición de equilibrio}
Ya hemos revisado que la primera condición de equilibrio se presenta cuando:
\pause
\begin{align*}
\nsum F_{x} &= 0 \\
\nsum F_{y} &= 0
\end{align*}
\end{frame}
\begin{frame}
\frametitle{Suma de las componentes en $x$}
Para la componente en $x$:
\pause
\begin{eqnarray*}
\begin{aligned}
\nsum F_{x} &= 0 \\ \pause
&= T_{1x} + T_{2x} + W_{x} = 0 \\ \pause
&(-0.766)(T_{1}) + (0.642)(T_{2}) + 0 = 0 \\  \pause
&(-0.766)(T_{1}) + (0.642)(T_{2}) = 0
\end{aligned}
\end{eqnarray*}
\end{frame}
\begin{frame}
\frametitle{Suma de las componentes en $y$}
Para la componente en $y$:
\pause
\begin{eqnarray*}
\begin{aligned}
\nsum F_{y} &= 0 \\ \pause
&= T_{1y} + T_{2y} + W_{y} = 0 \\ \pause
&(0.642)(T_{1}) + (0.766)(T_{2}) - \SI{200}{\newton} = 0
\end{aligned}
\end{eqnarray*}
\end{frame}
\begin{frame}
\frametitle{Sistema de ecuaciones}
Hemos obtenido el siguiente sistema simultáneo de ecuaciones:
\pause
\begin{align}
(-0.766)(T_{1}) + (0.642)(T_{2}) &= 0 \label{eq:ecuacion_08_01} \\[0.5em]
(0.642)(T_{1}) + (0.766)(T_{2}) - \SI{200}{\newton} &= 0 \label{eq:ecuacion_08_02}
\end{align}
\end{frame}
\begin{frame}
\frametitle{Resolviendo el sistema de ecuaciones}
De la ec. (\ref{eq:ecuacion_08_01}) despejamos $T_{2}$:
\pause
\begin{eqnarray*}
\begin{aligned}
(-0.766)(T_{1}) + (0.642)(T_{2}) &= 0 \\[0.5em] \pause
(0.642)(T_{2}) &= (0.766)(T_{1}) = \\[0.5em] \pause
T_{2} &= \dfrac{0.766}{(0.642)} \cdot T_{1} = \\[0.5em] \pause
T_{2} &= 1.193 \cdot T_{1}
\end{aligned}
\end{eqnarray*}
\end{frame}
\begin{frame}
\frametitle{Sustituyendo el valor}
Ahora ya podemos obtener el valor de la tensión $T_{1}$ usando la ec. (\ref{eq:ecuacion_08_02}):
\pause
\begin{eqnarray*}
\begin{aligned}
(0.642)(T_{1}) + (0.766)(T_{2}) - \SI{200}{\newton} &= 0 \\[0.5em] \pause
(0.642)(T_{1}) + (0.766)(1.193 \cdot T_{1}) - \SI{200}{\newton} &= 0 \\[0.5em] \pause
(0.642)(T_{1}) + (0.913)(T_{1}) - \SI{200}{\newton} &= 0 \\[0.5em] \pause
(1.555)(T_{1}) - \SI{200}{\newton} &= 0 \\[0.5em] \pause
(1.555)(T_{1}) &= \SI{200}{\newton}
\end{aligned}
\end{eqnarray*}
\end{frame}
\begin{frame}
\frametitle{Sustituyendo el valor}
\begin{eqnarray*}
\begin{aligned}
T_{1} &= \dfrac{\SI{200}{\newton}}{1.555} = \\[0.5em] \pause
T_{1} &= \SI{128.61}{\newton}
\end{aligned}
\end{eqnarray*}
\end{frame}
\begin{frame}
\frametitle{Recuperando el segundo valor}
Ahora podemos ocupar el valor de $T_{1}$ para obtener el valor de la tensión $T_{2}$, con la ec. (\ref{eq:ecuacion_08_01}):
\pause
\begin{eqnarray*}
\begin{aligned}
(-0.766)(T_{1}) + (0.642)(T_{2}) &= 0 \\[0.5em] \pause
(-0.766)(\SI{128.61}{\newton}) + (0.642)(T_{2}) &= 0 \\[0.5em] \pause
- \SI{98.51}{\newton} + (0.642)(T_{2}) &= 0 \\[0.5em] \pause
(0.642)(T_{2}) &= \SI{98.51}{\newton}
\end{aligned}
\end{eqnarray*}
\end{frame}
\begin{frame}
\frametitle{Recuperando el segundo valor}
\begin{eqnarray*}
\begin{aligned}
T_{2} &= \dfrac{\SI{98.51}{\newton}}{0.642} \\[0.5em] \pause
T_{2} &= \SI{153.44}{\newton}
\end{aligned}
\end{eqnarray*}
\end{frame}
\begin{frame}
\frametitle{Solución completa}
Hemos obtenido las tensiones en las cuerdas que soportan el anuncio:
\begin{align*}
T_{1} &= \SI{128.61}{\newton} \\[0.5em]
T_{2} &= \SI{153.44}{\newton}
\end{align*}
\end{frame}

\end{document}