\documentclass[12pt]{exam}
\usepackage[utf8]{inputenc}
\usepackage[T1]{fontenc}
\usepackage[spanish]{babel}
\usepackage[autostyle,spanish=mexican]{csquotes}
\usepackage{amsmath}
\usepackage{amsthm}
\usepackage{physics}
\usepackage{tikz}
\usepackage{float}
\usepackage[per-mode=symbol]{siunitx}
\usepackage{gensymb}
\usepackage{multicol}
\usepackage[left=2.00cm, right=2.00cm, top=2.00cm, 
     bottom=2.00cm]{geometry}

\usepackage[fontsize=14pt]{scrextend}
\usepackage{anyfontsize}

\renewcommand{\questionlabel}{\thequestion)}
\decimalpoint
\sisetup{bracket-numbers = false}


\title{\vspace*{-2cm}Diagnóstico de Repaso\vspace{-5ex}}
\date{\today}

\footer{}{\thepage}{}

\begin{document}
\maketitle

\section{Física.}

\begin{questions}
    \question Según el sistema internacional, las unidades fundamentales son:
    \\[0.5em]
    \begin{oneparchoices}
        \choice $2$
        \choice $4$
        \choice $5$
        \choice $7$
    \end{oneparchoices}
    \question Magnitud es:
    \begin{choices}
        \choice Algo que es magnífico.
        \choice Lo que se puede oler.
        \choice Lo que se puede ver.
        \choice Lo que se puede medir.
    \end{choices}
    \question La unidad fundamental de la longitud es el:
    \\[0.5em]
    \begin{oneparchoices}
        \choice Segundo.
        \choice Pulgadas.
        \choice Metro.
        \choice Litro.
    \end{oneparchoices}
    \question La unidad fundamental del tiempo es:
    \\[0.5em]
    \begin{oneparchoices}
        \choice Hora.
        \choice Kilogramo.
        \choice Segundo.        
        \choice Metro.
    \end{oneparchoices}
    \question ¿Qué relación de abreviaturas es correcta en el sistema internacional?
    \\[0.5em]
    I. \quad Segundo \quad seg \\
    II. \quad Mol \quad mol \\
    III. \quad Ampere\quad A 
    \\[0.5em]
    \begin{oneparchoices}
        \choice Sólo I.
        \choice Sólo II.
        \choice I y II.
        \choice II y III.    
    \end{oneparchoices}
    \question La unidad base de la intensidad luminosa es:
    \\[0.5em]
    \begin{oneparchoices}
        \choice Metro.
        \choice Candela.
        \choice Mol.
        \choice Segundo.
    \end{oneparchoices}
    \question Una unidad derivada es:
    \\[0.5em]
    \begin{oneparchoices}
        \choice Longitud.
        \choice Velocidad.
        \choice Tiempo.
        \choice Temperatura.
    \end{oneparchoices}
\end{questions}

\newpage

\section{Química.}

\textbf{Instrucciones}: Completa la frase con el(los) concepto(s) que hacen falta.

\begin{enumerate}
\item \rule{3cm}{0.1mm} Es la resistencia de un líquido a fluir.
\item \rule{3cm}{0.1mm} Es la relación de la masa de una sustancia por unidad de volumen.
\item \rule{3cm}{0.1mm} Se conoce como la capacidad de una sustancia para disolverse en otra. Se expresa en términos de la masa de una sustancia (\rule{3cm}{0.1mm}) que puede disolverse en una masa determinada de otra sustancia (\rule{3cm}{0.1mm}) a una temperatura dada.
\item \rule{3cm}{0.1mm} Mezcla homogénea, uniforme y estable, formada por dos o más sustancias denominadas genéricamente componentes.
\item Si se deja caer un balín de acero en vasos de precipitado que contienen diferentes líquidos, se obtienen los siguientes tiempos:
\begin{table}[H]
    \centering
    \begin{tabular}{c | c }
        Líquido & Tiempo (s) \\ \hline
        1 & $\num{2.54}$ \\ \hline
        2 & $\num{0.20}$ \\ \hline
        3 & $\num{1.18}$ \\ \hline
        4 & $\num{0.6}$ \\ \hline
    \end{tabular}
\end{table}
¿Qué líquido tiene mayor viscosidad? R = \rule{2cm}{0.1mm} \\
¿Qué líquido tiene menor viscosidad? R = \rule{2cm}{0.1mm}
\item Hablando de mezclas 1: Las \rule{3cm}{0.1mm} son aquellas en las que sus componenetes no están uniformemente distribuidos y conservan sus propiedades individuales.
\item Hablando de mezclas 2: Las \rule{3cm}{0.1mm} son aquellas en las que sus componentes están distribuidos uniformemente, sin poder distinguirlos.
% \item Hablando de mezclas 3: Las \rule{3cm}{0.1mm} pueden clasificarse como coloides y suspensiones.
% \item Hablando de mezclas 4: Definición de \rule{3cm}{0.1mm} : son mezclas formadas por un sólido en polvo o pequeñas partículas no solubles que se dispersan en un medio líquido denominado fase dispersa.
\end{enumerate}

% \section{Historia de México.}

% \subsection{Independencia.}

% En acuerdo con la línea de tiempo para la Independencia, relaciona las columnas de las ciudades o poblados mencionados en la lectura, con el correspondiente estado.
% \begin{table}[H]
%     \centering
%     \renewcommand{\arraystretch}{1.2}
%     \begin{tabular}{| l | p{5cm} | p{5cm} | } \hline
%         \textbf{Localidad} & \multicolumn{1}{c|}{\textbf{Estado}} & \multicolumn{1}{c|}{\textbf{Capital}} \\ \hline
%         Monte de las Cruces & & \\ \hline
%         Guadajalara & & \\ \hline
%         Norias de Baján & & \\ \hline
%         San Cristóbal Ecatepec & & \\ \hline
%         Soto la Marina & & \\ \hline
%         Córdoba & & \\ \hline
%     \end{tabular}
% \end{table}

% \subsection{Revolución mexicana.}

% En acuerdo con la línea de tiempo para la Revolución mexicana, relaciona las columnas de las ciudades o poblados mencionados en la lectura, con el correspondiente estado.
% \begin{table}[H]
%     \centering
%     \renewcommand{\arraystretch}{1.2}
%     \begin{tabular}{| l | p{5cm} | p{5cm} | } \hline
%         \textbf{Localidad} & \multicolumn{1}{c|}{\textbf{Estado}} & \multicolumn{1}{c|}{\textbf{Capital}} \\ \hline
%         Cananea & & \\ \hline
%         Orizaba & & \\ \hline
%         Monterrey & & \\ \hline
%         San Luis Potosí & & \\ \hline
%         Ciudad Juárez & & \\ \hline
%         Cuernavaca & & \\ \hline
%         Tampico & & \\ \hline
%         Celaya & & \\ \hline
%         Parral & & \\ \hline
%     \end{tabular}
% \end{table}

\end{document}