\documentclass[12pt]{beamer}
\usepackage{Estilos/BeamerFC}
\usepackage{Estilos/ColoresLatex}
\usetheme{Warsaw}
\usecolortheme{seahorse}
%\useoutertheme{default}
\setbeamercovered{invisible}
% or whatever (possibly just delete it)
\setbeamertemplate{section in toc}[sections numbered]
\setbeamertemplate{subsection in toc}[subsections numbered]
\setbeamertemplate{subsection in toc}{\leavevmode\leftskip=3.2em\rlap{\hskip-2em\inserttocsectionnumber.\inserttocsubsectionnumber}\inserttocsubsection\par}
\setbeamercolor{section in toc}{fg=blue}
\setbeamercolor{subsection in toc}{fg=blue}
\setbeamercolor{frametitle}{fg=blue}
\setbeamertemplate{caption}[numbered]

\setbeamertemplate{footline}
\beamertemplatenavigationsymbolsempty
\setbeamertemplate{headline}{}


\makeatletter
\setbeamercolor{section in foot}{bg=gray!30, fg=black!90!orange}
\setbeamercolor{subsection in foot}{bg=blue!30}
\setbeamercolor{date in foot}{bg=black}
\setbeamertemplate{footline}
{
  \leavevmode%
  \hbox{%
  \begin{beamercolorbox}[wd=.333333\paperwidth,ht=2.25ex,dp=1ex,center]{section in foot}%
    \usebeamerfont{section in foot} \insertsection
  \end{beamercolorbox}%
  \begin{beamercolorbox}[wd=.333333\paperwidth,ht=2.25ex,dp=1ex,center]{subsection in foot}%
    \usebeamerfont{subsection in foot}  \insertsubsection
  \end{beamercolorbox}%
  \begin{beamercolorbox}[wd=.333333\paperwidth,ht=2.25ex,dp=1ex,right]{date in head/foot}%
    \usebeamerfont{date in head/foot} {T1 - Segunda presentación} \hspace*{2em}
    \insertframenumber{} / \inserttotalframenumber \hspace*{2ex} 
  \end{beamercolorbox}}%
  \vskip0pt%
}
\makeatother

\makeatletter
\patchcmd{\beamer@sectionintoc}{\vskip1.5em}{\vskip0.8em}{}{}
\makeatother
\usepackage{pifont}
\newcommand{\cmark}{\ding{51}}%
\newcommand{\xmark}{\ding{55}}%

\makeatletter
\setbeamertemplate{footline}
{
  \leavevmode%
  \hbox{%
  \begin{beamercolorbox}[wd=.333333\paperwidth,ht=2.25ex,dp=1ex,center]{section in foot}%
    \usebeamerfont{section in foot} \insertsection
  \end{beamercolorbox}%
  \begin{beamercolorbox}[wd=.333333\paperwidth,ht=2.25ex,dp=1ex,center]{subsection in foot}%
    \usebeamerfont{subsection in foot}  \insertsubsection
  \end{beamercolorbox}%
  \begin{beamercolorbox}[wd=.333333\paperwidth,ht=2.25ex,dp=1ex,right]{date in head/foot}%
    \usebeamerfont{date in head/foot} \insertshortdate{} \hspace*{2em}
    \insertframenumber{} / \inserttotalframenumber \hspace*{2ex} 
  \end{beamercolorbox}}%
  \vskip0pt%
}
\makeatother

\sisetup{
  per-mode=fraction,
  fraction-function=\tfrac
}

\setbeamertemplate{navigation symbols}{}
\date{10 de abril}

% \sisetup{math-rm=\symup,detect-all}
\sisetup{detect-all, math-rm = \ensuremath}

\title{Sesión 6. Física}
\subtitle{Asesoría}

\begin{document}

\maketitle
\fontsize{14}{14}\selectfont
\spanishdecimal{.}

\section*{Contenido}
\frame[allowframebreaks]{\tableofcontents[currentsection, hideallsubsections]}


\section{Sistemas de vectores}
\frame{\tableofcontents[currentsection, hideothersubsections]}
\subsection{Ejercicio 1}

\begin{frame}
\frametitle{Enunciado del Ejercicio 1}
Del siguiente sistema de vectores determina por el método analítico el valor de la resultante y el ángulo que forma con respecto al eje $x$ positivo.
\end{frame}
\begin{frame}[plain]
\begin{figure}
\centering
\begin{tikzpicture}[scale=0.8]
    \draw (-4, 0) -- (9, 0) node [above, pos=1] {$x$};
    \draw (0, -4) -- (0, 6) node [left, pos=1] {$y$};
    \draw [-stealth, thick, color=blue] (0, 0) -- (8, 0) node [below, midway] {$\va{F}_{1} = \SI{8}{\newton}$};
    \draw [-stealth, thick, color=burgundy] (0, 0) -- (4.59, 3.85) node [above, near end, rotate=40] {$\va{F}_{2} = \SI{6}{\newton}$};
    \draw [color=burgundy] (0.5, 0) arc(0:40:0.5);
    \node [color=burgundy] at (2, 0.3) {$\theta_{2} = \ang{40}$};
    \draw [-stealth, thick, color=cerise] (0, 0) -- (-2.59, -1.5) node [below, near end, rotate=30] {$\va{F}_{3} = \SI{3}{\newton}$};
    \draw [color=cerise] (-0.5, 0) arc(180:210:0.5);
    \node [color=cerise] at (-2,- 0.3) {$\theta_{3} = \ang{30}$};
    \draw [-stealth, thick, color=byzantine] (0, 0) -- (0, 5) node [left, near end] {$\va{F}_{4} = \SI{5}{\newton}$};
\end{tikzpicture}
\end{figure}
\end{frame}
\begin{frame}
\frametitle{Recomendación para la solución}
Es conveniente enlistar los vectores involucrados de tal manera que tengamos tanto la magnitud como el ángulo que establece su dirección.
\\
\bigskip
\pause
Recordemos que todo vector debe de tener un ángulo asociado, veremos en particular que en el ejercicio hay dos vectores que coinciden con los ejes del sistema cartesiano.
\end{frame}
\begin{frame}
\frametitle{Lista de vectores}
\begin{table}
\centering
\begin{tabular}{c | c | c }
Vector & Magnitud (en $\unit{\newton}$) & Ángulo \\ \hline
$\va{F}_{1}$ & $8$ & $\theta_{1} = \ang{0}$ \\ \hline
$\va{F}_{2}$ & $6$ & $\theta_{2} = \ang{40}$ \\ \hline
$\va{F}_{3}$ & $3$ & $\theta_{3} = \ang{30}$ \\ \hline
$\va{F}_{4}$ & $5$ & $\theta_{4} = \ang{90}$ \\ \hline
\end{tabular}
\end{table}
\end{frame}
\begin{frame}
\frametitle{Calculando las componentes}
Una vez que tenemos la lista de vectores, iniciamos el cálculo de las componentes en las direcciones $x$ e $y$ de cada uno de ellos.
\\
\bigskip
\pause
Siendo recomendable también hacer una tabla con las componentes.
\end{frame}
\begin{frame}
\frametitle{Tabla con las componentes}
\begin{table}
\centering
\begin{tabular}{c | l | l | c}
Componente & Expresión & Sustitución & Valor \\ \hline
$F_{1x}$ & $\cos \theta_{1} \cdot F_{1}$ & $\cos \ang{0} \cdot \SI{8}{\newton}$ & $\SI{8}{\newton}$ \\ \hline
$F_{1y}$ & $\sin \theta_{1} \cdot F_{1}$ & $\sin \ang{0} \cdot \SI{8}{\newton}$ & $\SI{0}{\newton}$ \\ \hline
$F_{2x}$ & $\cos \theta_{2} \cdot F_{2}$ & $\cos \ang{40} \cdot \SI{6}{\newton}$ & $\SI{4.59}{\newton}$ \\ \hline
$F_{2y}$ & $\sin \theta_{2} \cdot F_{2}$ & $\sin \ang{40} \cdot \SI{6}{\newton}$ & $\SI{3.85}{\newton}$ \\ \hline
\end{tabular}
\end{table}
\end{frame}
\begin{frame}
\frametitle{Tabla con las componentes}
\begin{table}
\centering
\begin{tabular}{c | l | l | c}
Componente & Expresión & Sustitución & Valor \\ \hline
$F_{3x}$ & $-\cos \theta_{3} \cdot F_{3}$ & $-\cos \ang{30} \cdot \SI{3}{\newton}$ & $-\SI{2.59}{\newton}$ \\ \hline
$F_{3y}$ & $-\sin \theta_{3} \cdot F_{3}$ & $-\sin \ang{30} \cdot \SI{3}{\newton}$ & $-\SI{1.5}{\newton}$ \\ \hline
$F_{4x}$ & $\cos \theta_{4} \cdot F_{4}$ & $\cos \ang{90} \cdot \SI{5}{\newton}$ & $\SI{0}{\newton}$ \\ \hline
$F_{4y}$ & $\sin \theta_{4} \cdot F_{4}$ & $\sin \ang{90} \cdot \SI{5}{\newton}$ & $\SI{5}{\newton}$ \\ \hline
\end{tabular}
\end{table}
\end{frame}
\begin{frame}
\frametitle{Recuperando las componentes del vector resultante}
Ahora ya podemos calcular las componentes del vector resultante, \pause recordemos que:
\pause
\begin{eqnarray*}
\begin{aligned}
F_{Rx} &= \nsum_{i=1}^{4} F_{ix} = \\[0.5em] \pause
&= F_{1x} + F_{2x} + F_{3x} + F_{4x} = \\[0.5em] \pause
&= \SI{8}{\newton} {+} \SI{4.59}{\newton} {+} (-\SI{2.59}{\newton}) {+} \SI{0}{\newton} = \\[0.5em] \pause
&= \SI{9.69}{\newton}
\end{aligned}
\end{eqnarray*}
\end{frame}
\begin{frame}
\frametitle{Recuperando las componentes del vector resultante}
Para la componente $F_{Ry}$:
\pause
\begin{eqnarray*}
\begin{aligned}
F_{Ry} &= \nsum_{i=1}^{4} F_{iy} = \\[0.5em] \pause
&= F_{1y} + F_{2y} + F_{3y} + F_{4y} = \\[0.5em] \pause
&= \SI{0}{\newton} {+} \SI{3.85}{\newton} {+} (-\SI{1.5}{\newton}) {+} \SI{5}{\newton} = \\[0.5em] \pause
&= \SI{7.35}{\newton}
\end{aligned}
\end{eqnarray*}
\end{frame}
\begin{frame}
\frametitle{Magnitud del vector resultante}
Una vez obtenidas las componentes en $x$ e $y$, calculamos la magnitud del vector resultante:
\pause
\begin{eqnarray*}
\begin{aligned}
\abs{F_{R}} &= \sqrt{(F_{Rx})^{2} + (F_{Ry})^{2}} = \\[0.5em] \pause
&= \sqrt{(\SI{9.69}{\newton})^{2} + (\SI{7.35}{\newton})^{2}} = \\[0.5em] \pause
&= \sqrt{\SI{93.89}{\square\newton} + \SI{54.02}{\square\newton}} = \\[0.5em] \pause
&= \sqrt{\SI{147.91}{\square\newton}} = \\[0.5em] \pause
&= \SI{12.16}{\newton}
\end{aligned}
\end{eqnarray*}
\end{frame}
\begin{frame}
\frametitle{El ángulo del vector resultante}
Tenemos que $F_{Rx} > 0$ y $F_{Ry} > 0$, por lo que el vector resultante está en el cuadrante I, \pause con esos valores podemos calcular el valor del ángulo del vector resultante con respecto al eje $x$ positivo:
\pause
\begin{eqnarray*}
\begin{aligned}
\tan \theta_{R} &= \dfrac{F_{Ry}}{F_{Rx}} = \pause
 \dfrac{\SI{7.35}{\newton}}{\SI{9.69}{\newton}} = \pause  0.7585 \\[0.5em] \pause
\arctan(\tan \theta_{R}) &= \arctan(0.7585) \\[0.5em] \pause
\theta_{R} &= \ang{37.18}
\end{aligned}
\end{eqnarray*}
\end{frame}
\begin{frame}[plain]
\begin{figure}
\centering
\begin{tikzpicture}[scale=0.5]
    \draw (-4, 0) -- (9, 0) node [above, pos=1] {$x$};
    \draw (0, -4) -- (0, 6) node [left, pos=1] {$y$};
    \draw [-stealth, thick, color=blue] (0, 0) -- (8, 0) node [below, midway] {\small{$\va{F}_{1}$}};

    \draw [-stealth, thick, color=burgundy] (0, 0) -- (4.59, 3.85) node [above, near end, rotate=40] {\small{$\va{F}_{2}$}};
    \draw [color=burgundy] (0.5, 0) arc(0:40:0.5);
    \node [color=burgundy] at (1.6, 0.4) {\small{$\theta_{2}$}};

    \draw [-stealth, thick, color=cerise] (0, 0) -- (-2.59, -1.5) node [below, near end, rotate=30] {\small{$\va{F}_{3}$}};
    \draw [color=cerise] (-0.5, 0) arc(180:210:0.5);
    \node [color=cerise] at (-1.8,- 0.4) {\small{$\theta_{3}$}};

    \draw [-stealth, thick, color=byzantine] (0, 0) -- (0, 5) node [left, near end] {\small{$\va{F}_{4}$}};
    
    \draw [-stealth, thick, color=cadmiumred] (0, 0) -- (9.69, 7.35) node [above, near end, rotate=37] {\small{$\va{F}_{R} = \SI{12.16}{\newton}$}};
    \draw [thick, color=cadmiumred] (3, 0) arc(0:37:3);
    \node [color=cadmiumred] at (5, 0.8) {\small{$\theta_{R} = \ang{37.18}$}};

\end{tikzpicture}
\end{figure}
\end{frame}

\subsection{Ejercicio 3 Guía}

\begin{frame}
\frametitle{Enunciado del Ejercicio 3}
Realiza la suma de los vectores, determinando el vector resultante y el ángulo que forma la resultante con respecto al eje $x$ positivo.
\end{frame}
\begin{frame}
\frametitle{Diagrama para el Ejercicio 3}
\begin{figure}
\centering
\begin{tikzpicture}
  \draw (-6, 0) -- (2, 0);
  \draw (0, -2) -- (0, 1.5);
  \draw [-stealth, thick, color=ao] (0, 0) -- (1.69, 0.61) node [above, pos=1.2] {\small{$T_{1} = \SI{18}{\newton}$}};
  \draw [color=ao] (0.5, 0) arc(0:20:0.4);
  \node at (2, 0.2) [color=ao] {\small{$\theta_{1} = \ang{20}$}};

  \draw [-stealth, thick, color=darkmagenta] (0, 0) -- (-5.6, 0) node [above, near end] {\small{$T_{3} = \SI{58}{\newton}$}};
  \draw [color=darkmagenta] (0.4, 0) arc(0:180:0.5);
  \node at (-1.5, 0.4) [color=darkmagenta] {\small{$\theta_{3} = \ang{180}$}};
  
  \draw [-stealth, thick, color=officegreen] (0, 0) -- (0.306, -0.629) node [above, xshift=0.5cm, yshift=-0.7cm] {\small{$T_{2} = \SI{7}{\newton}$}};
  \draw [color=officegreen] (0.3, 0) arc(360:296:0.3);
  \node at (1.3, -0.4) [color=officegreen] {\small{$\theta_{2} = \ang{64}$}};
\end{tikzpicture}
\end{figure}
\end{frame}
\begin{frame}
\frametitle{Tabla con las componentes}
\begin{table}
\centering
\begin{tabular}{c | l | l | c}
Componente & Expresión & Sustitución & Valor \\ \hline
$T_{1x}$ & $\cos \theta_{1} \cdot T_{1}$ & $\cos \ang{20} \cdot \SI{18}{\newton}$ & $\SI{16.91}{\newton}$ \\ \hline
$T_{1y}$ & $\sin \theta_{1} \cdot T_{1}$ & $\sin \ang{20} \cdot \SI{18}{\newton}$ & $\SI{6.15}{\newton}$ \\ \hline
$T_{2x}$ & $\cos \theta_{2} \cdot T_{2}$ & $\cos \ang{64} \cdot \SI{7}{\newton}$ & $\SI{3.06}{\newton}$ \\ \hline
$T_{2y}$ & $-\sin \theta_{2} \cdot T_{2}$ & $-\sin \ang{64} \cdot \SI{7}{\newton}$ & $-\SI{6.29}{\newton}$ \\ \hline
\end{tabular}
\end{table}
\end{frame}
\begin{frame}
  \frametitle{Tabla con las componentes}
  \begin{table}
  \centering
\begin{tabular}{c | l | l | c}
Componente & Expresión & Sustitución & Valor \\ \hline
$T_{3x}$ & $\cos \theta_{3} \cdot T_{3}$ & $\cos \ang{180} \cdot \SI{58}{\newton}$ & $-\SI{58}{\newton}$ \\ \hline
$T_{3y}$ & $\sin \theta_{3} \cdot T_{3}$ & $\sin \ang{180} \cdot \SI{58}{\newton}$ & $\SI{0}{\newton}$ \\ \hline
\end{tabular}
\end{table}
\end{frame}
\begin{frame}
\frametitle{Sumando las componentes}
Obtenemos las componentes del vector resultante $T_{R}$ al sumar las componentes tanto en $x$ como en $y$.
\end{frame}
\begin{frame}
\frametitle{Componente en $x$}
Sabemos que:
\pause
\begin{eqnarray*}
\begin{aligned}
T_{Rx} &= \nsum_{i=1}^{3} T_{ix} = \\[0.5em] \pause
&= T_{1x} + T_{2x} + T_{3x} = \\[0.5em] \pause
&= \SI{16.91}{\newton} + \SI{3.06}{\newton} + (-\SI{58}{\newton}) = \\[0.5em] \pause
&= - \SI{38.03}{\newton}
\end{aligned}
\end{eqnarray*}
\end{frame}
\begin{frame}
\frametitle{Componente en $y$}
Sabemos que:
\pause
\begin{eqnarray*}
\begin{aligned}
T_{Ry} &= \nsum_{i=1}^{3} T_{iy} = \\[0.5em] \pause
&= T_{1y} + T_{2y} + T_{3y} = \\[0.5em] \pause
&= \SI{6.15}{\newton} + (-\SI{6.29}{\newton}) + \SI{0}{\newton} = \\[0.5em] \pause
&= -\SI{0.17}{\newton}
\end{aligned}
\end{eqnarray*}
\end{frame}
\begin{frame}
\frametitle{Calculando la magnitud del vector resultante}
Una vez conocidas las componentes $T_{Rx}$ y $T_{Ry}$, podemos obtener la magnitud del vector resultante:
\pause
\begin{eqnarray*}
\begin{aligned}
\abs{T_{R}} &= \sqrt{(T_{Rx})^{2} + (T_{Ry})^{2}} = \\[0.5em] \pause
&= \sqrt{ (- \SI{38.03}{\newton})^{2} + (-\SI{0.17}{\newton})^{2}} = \\[0.5em] \pause
&= \sqrt{ \SI{1146.28}{\square\newton} + \SI{0.0289}{\square\newton}} = \\[0.5em] \pause
&= \sqrt{\SI{1446.30}{\square\newton}} = \\[0.5em] \pause
&= \SI{38.03}{\newton}
\end{aligned}
\end{eqnarray*}
\end{frame}
\begin{frame}
\frametitle{El ángulo del vector resultante}
Revisando que las componentes $T_{Rx} < 0$ y $T_{Ry} < 0$, por lo que el vector resultante está en el cuadrante III, \pause con esos valores podemos calcular el valor del ángulo auxiliar $\alpha$ del vector resultante:
\pause
\begin{eqnarray*}
\begin{aligned}
\tan \alpha &= - \dfrac{T_{Ry}}{T_{Rx}} = \pause
 - \dfrac{\SI{0.17}{\newton}}{\SI{38.03}{\newton}} = \pause - 0.00447 \\[0.5em] \pause
\arctan(\tan \alpha) &= \arctan(- 0.00447) \\[0.5em] \pause
\alpha &= - \ang{0.25}
\end{aligned}
\end{eqnarray*}
\end{frame}
\begin{frame}
\frametitle{El ángulo del vector resultante}
Como el vector resultante está en el cuadrante III, ya tenemos un recorrido de \ang{180}, por lo que el ángulo $\theta_{R}$ se obtiene al sumar el valor de $\alpha$ a \ang{180}:
\pause
\begin{eqnarray*}
\begin{aligned}
\theta_{R} &= \ang{180} + \alpha = \pause \ang{180} + \ang{0.25} = \\[0.5em] \pause
\theta_{R} &= \ang{180.25}
\end{aligned}
\end{eqnarray*}
\end{frame}
\begin{frame}
\frametitle{Solución al sistema de vectores}
\begin{figure}
\centering
\begin{tikzpicture}[scale=1.3]
  \draw (-6, 0) -- (2, 0);
  \draw (0, -1) -- (0, 1.5);
  \draw [-stealth, thick, color=ao] (0, 0) -- (1.69, 0.61) node [above, pos=1.2] {\small{$T_{1}$}};
  \draw [color=ao] (0.5, 0) arc(0:20:0.4);
  \node at (1.2, 0.2) [color=ao] {\small{$\theta_{1}$}};
  \draw [-stealth, color=ao] (1, 0.2) -- (0.5, 0.1);

  \draw [-stealth, thick, color=darkmagenta] (0, 0) -- (-5.6, 0) node [above, near end] {\small{$T_{3}$}};
  \draw [color=darkmagenta] (0.4, 0) arc(0:180:0.5);
  \node at (-0.4, -0.7) [color=darkmagenta] {\small{$\theta_{3}$}};
  \draw [-stealth, color=darkmagenta](-0.4, -0.5) -- (-0.2, 0.5);
  
  \draw [-stealth, thick, color=officegreen] (0, 0) -- (0.306, -0.629) node [above, xshift=0.5cm, yshift=-0.7cm] {\small{$T_{2}$}};
  \draw [color=officegreen] (0.3, 0) arc(360:296:0.3);
  \node at (0.5, -0.3) [color=officegreen] {\small{$\theta_{2}$}};

  \draw [-stealth, thick, color=orange-red] (0, 0) -- (-3.8, 0.017) node [above, near end] {\small{$T_{R} = \SI{38.03}{\newton}$}};
  \draw [thick, color=orange-red] (0.9, 0) arc(0:180.25:0.9);
  \node at (-1, 1.2) [color=orange-red] {\small{$\theta_{R} = \ang{180.25}$}};

  % \draw [thick, color=black] (-0.9, 0) arc(180:161:0.9);
  % \node at (-1.3, 0.2) [color=black] {\small{$\alpha$}};
\end{tikzpicture}
\end{figure}
\end{frame}

\end{document}