\documentclass[12pt]{beamer}
\usepackage{Estilos/BeamerFC}
\usepackage{Estilos/ColoresLatex}
\usetheme{Warsaw}
\usecolortheme{seahorse}
%\useoutertheme{default}
\setbeamercovered{invisible}
% or whatever (possibly just delete it)
\setbeamertemplate{section in toc}[sections numbered]
\setbeamertemplate{subsection in toc}[subsections numbered]
\setbeamertemplate{subsection in toc}{\leavevmode\leftskip=3.2em\rlap{\hskip-2em\inserttocsectionnumber.\inserttocsubsectionnumber}\inserttocsubsection\par}
\setbeamercolor{section in toc}{fg=blue}
\setbeamercolor{subsection in toc}{fg=blue}
\setbeamercolor{frametitle}{fg=blue}
\setbeamertemplate{caption}[numbered]

\setbeamertemplate{footline}
\beamertemplatenavigationsymbolsempty
\setbeamertemplate{headline}{}


\makeatletter
\setbeamercolor{section in foot}{bg=gray!30, fg=black!90!orange}
\setbeamercolor{subsection in foot}{bg=blue!30}
\setbeamercolor{date in foot}{bg=black}
\setbeamertemplate{footline}
{
  \leavevmode%
  \hbox{%
  \begin{beamercolorbox}[wd=.333333\paperwidth,ht=2.25ex,dp=1ex,center]{section in foot}%
    \usebeamerfont{section in foot} \insertsection
  \end{beamercolorbox}%
  \begin{beamercolorbox}[wd=.333333\paperwidth,ht=2.25ex,dp=1ex,center]{subsection in foot}%
    \usebeamerfont{subsection in foot}  \insertsubsection
  \end{beamercolorbox}%
  \begin{beamercolorbox}[wd=.333333\paperwidth,ht=2.25ex,dp=1ex,right]{date in head/foot}%
    \usebeamerfont{date in head/foot} {T1 - Segunda presentación} \hspace*{2em}
    \insertframenumber{} / \inserttotalframenumber \hspace*{2ex} 
  \end{beamercolorbox}}%
  \vskip0pt%
}
\makeatother

\makeatletter
\patchcmd{\beamer@sectionintoc}{\vskip1.5em}{\vskip0.8em}{}{}
\makeatother
\usepackage{pifont}
\newcommand{\cmark}{\ding{51}}%
\newcommand{\xmark}{\ding{55}}%

\makeatletter
\setbeamertemplate{footline}
{
  \leavevmode%
  \hbox{%
  \begin{beamercolorbox}[wd=.333333\paperwidth,ht=2.25ex,dp=1ex,center]{section in foot}%
    \usebeamerfont{section in foot} \insertsection
  \end{beamercolorbox}%
  \begin{beamercolorbox}[wd=.333333\paperwidth,ht=2.25ex,dp=1ex,center]{subsection in foot}%
    \usebeamerfont{subsection in foot}  \insertsubsection
  \end{beamercolorbox}%
  \begin{beamercolorbox}[wd=.333333\paperwidth,ht=2.25ex,dp=1ex,right]{date in head/foot}%
    \usebeamerfont{date in head/foot} \insertshortdate{} \hspace*{2em}
    \insertframenumber{} / \inserttotalframenumber \hspace*{2ex} 
  \end{beamercolorbox}}%
  \vskip0pt%
}
\makeatother

\setbeamertemplate{navigation symbols}{}
\date{23 de marzo}

\title{Sesión 5. Física}
\subtitle{Asesoría}

\begin{document}

\maketitle
\fontsize{14}{14}\selectfont
\spanishdecimal{.}

\section*{Contenido}
\frame[allowframebreaks]{\tableofcontents[currentsection, hideallsubsections]}


\section{Resolviendo el Ejercicio 2}
\frame{\tableofcontents[currentsection, hideothersubsections]}
\subsection{El problema a resolver}

\begin{frame}
\frametitle{Enunciado del Ejercicio 2}
En una persecución policial, el automóvil en fuga inicia con los siguientes desplazamientos:
\setbeamercolor{item projected}{bg=lava,fg=white}
\setbeamertemplate{enumerate items}{%
\usebeamercolor[bg]{item projected}%
\raisebox{1.5pt}{\colorbox{bg}{\color{fg}\footnotesize\insertenumlabel}}%
}
\begin{enumerate}[<+->]
\item $d_{1} = \SI{120}{\meter}, \theta = \ang{10}$
\item $d_{2} = \SI{900}{\meter}, \theta = \ang{100}$
\item $d_{3} = \SI{700}{\meter}, \theta = \ang{200}$
\item $d_{4} = \SI{500}{\meter}, \theta = \ang{300}$
\end{enumerate}
\end{frame}
\begin{frame}
\frametitle{Enunciado del Ejercicio 2}
Si una patrulla está en la posición donde inició el automóvil la fuga.
\\
\bigskip
\pause
¿Cuál es el desplazamiento que tiene que realizar la patrulla par encontrarse con el automóvil que se dio a la fuga?
\\
\bigskip
\pause
Determina el valor del desplazamiento resultante, con respecto al eje $x$ positivo.
\end{frame}
\begin{frame}
\frametitle{Método gráfico}
Nuevamente ocupamos el método gráfico para encontrar el vector resultante.
\\
\bigskip
\pause
A continuación se revisan los desplazamientos del automóvil en fuga.
\end{frame}
\begin{frame}[plain]
\begin{figure}
    \centering
    \begin{tikzpicture}[scale=0.8]
        % \draw[thin, gray!40] (-7, 0) grid (5, 9);
        \draw [-stealth, thick] (0, 0) -- (1.181, 0.208) node [above, near end] {$d_{1}$};
        \draw (0, 0) -- (1, 0);
        \draw (0.5, 0) arc(0:10:0.4);
        \node at (1.5, -0.3) {$\theta_{1} = \ang{10}$};
        \pause
        \draw [-stealth, thick] (1.181, 0.208) -- (-0.381, 9.071) node [right, midway] {$d_{2}$};
        \draw (1.181, 0.208) -- (2.181, 0.208);
        \draw (1.681, 0.208) arc(0:100:0.5);
        \node at (3.1, 0.5) {$\theta_{2} = \ang{100}$};
        \pause
        \draw [-stealth, thick] (-0.381, 9.071) -- (-6.958, 6.677) node [above, midway] {$d_{3}$};
        \draw (-0.381, 9.071) -- (0.6, 9.071);
        \draw (0.3, 9.071) arc(0:200:0.5);
        \node at (1.4, 9.5) {$\theta_{3} = \ang{200}$};
        \pause
        \draw [-stealth, thick] (-6.958, 6.677) -- (-2.655, 4.177) node [above, midway] {$d_{4}$};
        \draw (-6.958, 6.677) -- (-5.958, 6.677);
        \draw (-6.3, 6.677) arc(0:300:0.4);
        \node at (-6.5, 7.8) {$\theta_{4} = \ang{300}$};
        \pause
        \draw [-stealth, thick, color=blue] (0, 0) -- (-2.655, 4.177) node [below, midway] {$d_{R}$};
        \pause
        \draw [thick, color=blue] (0.4, 0) arc(0:130:0.38);
        \node at (-1, 0.3) [color=blue] {$\theta_{R}$};
    \end{tikzpicture}
\end{figure}
\end{frame}
\begin{frame}
\frametitle{Solución analítica}
Repetimos el procedimiento para obtener la solución analítica mediante la suma de las componentes de cada vector en las direcciones $x$ e $y$.
\end{frame}
\begin{frame}
\frametitle{Simplificando el problema}
Para resolver de manera más ágil el ejercicio, \pause a partir del ángulo del vector que se indica, identificamos el cuadrante al que pertenece.
\\
\bigskip
\pause
De esta manera tendremos las correspondientes expresiones para calcular las componentes de cada uno de ellos tanto en la dirección $x$ como en $y$.
\end{frame}
\begin{frame}
\frametitle{Componentes del vector $d_{1}$}
Como $\theta_{1} = \ang{10} < \ang{90}$
el vector $d_{1}$ se encuentra en el cuadrante I.
\\
\bigskip
\pause
Por lo tanto:
\pause
\begin{eqnarray*}
\begin{aligned}
d_{1x} &= \cos \theta_{1} \cdot d_{1} \pause = \cos \ang{10} \, \cdot \SI{120}{\meter} = \pause \SI{118.17}{\meter} \\[0.5em] \pause
d_{1y} &= \sin \theta_{1} \cdot d_{1} \pause =\sin \ang{10} \, \cdot \SI{120}{\meter} = \pause \SI{20.83}{\meter}
\end{aligned}
\end{eqnarray*}
\end{frame}
\begin{frame}
\frametitle{Componentes del vector $d_{2}$}
Para el vector $d_{2}$ revisamos que el ángulo:
\pause
\begin{eqnarray*}
\ang{90} < \theta_{2} = \ang{100} < \ang{180}
\end{eqnarray*}
\pause
por lo que el vector $d_{2}$ se encuentra en el cuadrante II.
\end{frame}
\begin{frame}
\frametitle{Componentes del vector $d_{2}$}
Tenemos que ocupar un ángulo auxiliar $\alpha$, que es el ángulo suplementario, es decir:
\pause
\begin{align*}
\theta = \ang{180} - \alpha
\end{align*}
\pause
Siendo entonces:
\pause
\begin{eqnarray*}
\alpha = \ang{180} - \ang{100} = \pause \ang{80}
\end{eqnarray*}
\end{frame}
\begin{frame}
\frametitle{Componentes del vector $d_{2}$}
Entonces las componentes del vector $d_{2}$ son:
\pause
\begin{eqnarray*}
\begin{aligned}
d_{2x} &= -\cos \alpha \cdot d_{2} \pause = -\cos \ang{80} \, \cdot \SI{900}{\meter} = \pause -\SI{156.28}{\meter} \\[0.5em] \pause
d_{2y} &= \sin \alpha \cdot d_{2} \pause = \sin \ang{80} \, \cdot \SI{900}{\meter} = \pause \SI{886.32}{\meter}
\end{aligned}
\end{eqnarray*}
\end{frame}
\begin{frame}
\frametitle{Componentes del vector $d_{3}$}
Para el vector $d_{3}$ vemos que el ángulo:
\pause
\begin{eqnarray*}
\ang{180} < \theta_{3} = \ang{200} < \ang{270}
\end{eqnarray*}
\pause
por lo que el vector $d_{3}$ se encuentra en el cuadrante III.
\end{frame}
\begin{frame}
\frametitle{Componentes del vector $d_{3}$}
Nuevamente tenemos que ocupar un ángulo auxiliar $\beta$, que es el ángulo complementario en el cuadrante III, es decir:
\pause
\begin{align*}
\theta = \ang{270} - \beta
\end{align*}
\pause
Siendo entonces:
\pause
\begin{eqnarray*}
\beta = \ang{270} - \ang{200} = \pause \ang{70}
\end{eqnarray*}
\end{frame}
\begin{frame}
\frametitle{Componentes del vector $d_{3}$}
Entonces las componentes del vector $d_{3}$ son:
\pause
\begin{eqnarray*}
\begin{aligned}
d_{3x} &= -\sin \beta \cdot d_{3} \pause = -\sin \ang{70} \, \cdot \SI{700}{\meter} = \pause -\SI{657.78}{\meter} \\[0.5em] \pause
d_{3y} &= -\cos \beta \cdot d_{3} \pause = -\cos \ang{70} \, \cdot \SI{700}{\meter} = \pause -\SI{239.41}{\meter}
\end{aligned}
\end{eqnarray*}
\end{frame}
\begin{frame}
\frametitle{Componentes del vector $d_{4}$}
Para el último vector $d_{4}$ vemos que el ángulo:
\pause
\begin{eqnarray*}
\ang{270} < \theta_{4} = \ang{300} < \ang{360}
\end{eqnarray*}
\pause
por lo que el vector $d_{4}$ se encuentra en el cuadrante IV.
\end{frame}
\begin{frame}
\frametitle{Componentes del vector $d_{4}$}
Nuevamente tenemos que ocupar un ángulo auxiliar $\psi$, que es el ángulo suplementario en el cuadrante IV, es decir:
\pause
\begin{align*}
\theta = \ang{360} - \psi
\end{align*}
\pause
Siendo entonces:
\pause
\begin{eqnarray*}
\psi = \ang{360} - \ang{300} = \pause \ang{60}
\end{eqnarray*}
\end{frame}
\begin{frame}
\frametitle{Componentes del vector $d_{4}$}
Entonces las componentes del vector $d_{4}$ son:
\pause
\begin{eqnarray*}
\begin{aligned}
d_{4x} &= \cos \psi \cdot d_{4} \pause = \cos \ang{60} \, \cdot \SI{500}{\meter} = \pause \SI{250}{\meter} \\[0.5em] \pause
d_{4y} &= -\sin \psi \cdot d_{4} \pause = - \sin \ang{60} \, \cdot \SI{500}{\meter} = \pause = -\SI{433.01}{\meter}
\end{aligned}
\end{eqnarray*}
\end{frame}
\begin{frame}
\frametitle{Sumando las componentes}
Una vez que ya hemos calculado las componentes en de los vectores en las direcciones $x$, $y$, ahora pasamos a sumar el valor de cada una de ellas para obtener las componentes del vector resultante.
\end{frame}
\begin{frame}
\frametitle{Componentes del vector resultante}
Para la componente $d_{Rx}$ se tiene entonces que:
\pause
\begin{eqnarray*}
\begin{aligned}
d_{Rx} &= \nsum_{n=1}^{4} d_{ix} = \pause d_{1x} + d_{2x} + d_{3x} + d_{4x} = \\[0.5em] \pause
&= \SI{118.17}{\meter} {+} (-\SI{156.28}{\meter}) {+} (-\SI{657.78}{\meter}) {+} \SI{250}{\meter} = \\[0.5em] \pause
&= -\SI{445.89}{\meter}
\end{aligned}
\end{eqnarray*}
\end{frame}
\begin{frame}
\frametitle{Componentes del vector resultante}
Ahora para la componente $d_{Ry}$ se tiene que:
\pause
\begin{eqnarray*}
\begin{aligned}
d_{Ry} &= \nsum_{n=1}^{4} d_{iy} = \pause d_{1y} + d_{2y} + d_{3y} + d_{4y} = \\[0.5em] \pause
&= \SI{20.83}{\meter} {+} \SI{886.32}{\meter} {+} (-\SI{239.41}{\meter}) {+} (-\SI{433.01}{\meter}) = \\[0.5em] \pause
&= \SI{234.73}{\meter}
\end{aligned}
\end{eqnarray*}
\end{frame}
\begin{frame}
\frametitle{Magnitude del vector resultante}
Una vez obtenidas las componentes del vector resultante, podemos obtener su magnitud:
\pause
\begin{eqnarray*}
\begin{aligned}
\abs{d_{R}} &= \sqrt{ (d_{Rx})^{2} + (d_{Ry})^{2} } = \\[0.5em] \pause
&= \sqrt{(-\SI{445.89}{\meter})^{2} + (\SI{234.73}{\meter})^{2}} = \\[0.5em] \pause
&= \sqrt{ \SI{198817.89}{\square\meter} + \SI{55098.17}{\square\meter} } = \\[0.5em] \pause
&= \sqrt{\SI{253916.06}{\square\meter}} = \\[0.5em] \pause
&= \SI{503.90}{\meter}
\end{aligned}
\end{eqnarray*}
\end{frame}
\begin{frame}
\frametitle{El ángulo del vector resultante}
De los valores de las componentes $d_{Rx}$, $d_{Ry}$ podemos reconocer el cuadrante en el que se encuentra el vector $d_{R}$:
\pause
\begin{align*}
d_{Rx} &= -\SI{445.89}{\meter} \\[0.5em]
d_{Ry} &=  \SI{234.73}{\meter}
\end{align*}
\pause
Entonces, el vector resultante está en el cuadrante II.
\end{frame}
\begin{frame}
\frametitle{El ángulo del vector resultante}
Ocuparemos un ángulo auxiliar $\alpha$, de tal manera que:
\pause
\begin{eqnarray*}
\begin{aligned}
\tan \alpha &= - \dfrac{d_{Ry}}{d_{Rx}} = \\[0.5em] \pause
\tan \alpha &= - \dfrac{\SI{445.89}{\meter}}{\SI{234.73}{\meter}} = \\[0.5em] \pause
\tan \alpha &= - 1.8995
\end{aligned}
\end{eqnarray*}
\end{frame}
\begin{frame}
\frametitle{El ángulo del vector resultante}
Pero como queremos el valor del ángulo, ocupamos la función inversa, el arco tangente:
\pause
\begin{eqnarray*}
\begin{aligned}
\arctan(\tan \alpha) &= \arctan(-1.8995) = \\[0.5em] \pause
\alpha &= -\ang{62.23}
\end{aligned}
\end{eqnarray*}
\pause
El signo negativo nos indica que se está midiendo desde el eje $x$ negativo, en el sentido horario del reloj.
\end{frame}
\begin{frame}
\frametitle{El ángulo del vector resultante}
Como sabemos que $\alpha$ es ángulo suplementario, entonces ocupamos su valor positivo:
\pause
\begin{eqnarray*}
\begin{aligned}
\theta_{R} &= \ang{180} - \alpha = \pause \ang{180} - \ang{62.23} = \\[0.5em] \pause
\theta_{R} &= \ang{117.77}
\end{aligned}
\end{eqnarray*}
\pause
De esta manera se ha resuelto el ejercicio 2 de la guía.
\end{frame}

\section{Ejercicios}
\frame{\tableofcontents[currentsection, hideothersubsections]}
\subsection{Ejercicio 1}

\begin{frame}
\frametitle{Enunciado del Ejercicio 1}
Del siguiente sistema de vectores determina por el método analítico el valor de la resultante y el ángulo que forma con respecto al eje $x$ positivo.
\end{frame}
\begin{frame}[plain]
\begin{figure}
\centering
\begin{tikzpicture}[scale=0.8]
    \draw (-4, 0) -- (9, 0);
    \draw (0, -4) -- (0, 6);
    \draw [-stealth, thick, color=blue] (0, 0) -- (8, 0) node [below, midway] {$\va{F}_{1} = \SI{8}{\newton}$};
    \draw [-stealth, thick, color=blue] (0, 0) -- (4.59, 3.85) node [above, midway, rotate=40] {$\va{F}_{2} = \SI{6}{\newton}$};
    \draw [color=blue] (0.5, 0) arc(0:40:0.5);
    \node [color=blue] at (2, 0.3) {$\theta_{2} = \ang{40}$};
    \draw [-stealth, thick, color=blue] (0, 0) -- (-2.59, -1.5) node [below, midway, rotate=30] {$\va{F}_{3} = \SI{3}{\newton}$};
    \draw [color=blue] (-0.5, 0) arc(180:210:0.5);
    \node [color=blue] at (-2,- 0.3) {$\theta_{3} = \ang{30}$};
    \draw [-stealth, thick, color=blue] (0, 0) -- (0, 5) node [left, midway] {$\va{F}_{4} = \SI{5}{\newton}$};
\end{tikzpicture}
\end{figure}
\end{frame}
\begin{frame}
\frametitle{Recomendación para la solución}
Es conveniente enlistar los vectores involucrados de tal manera que tengamos tanto la magnitud como el ángulo que establece su dirección.
\end{frame}
\begin{frame}
\frametitle{Lista de vectores}
\begin{table}
\centering
\begin{tabular}{c | c | c }
Vector & Magnitud (en $\unit{\newton}$) & Ángulo \\ \hline
$\va{F}_{1}$ & $8$ & $\theta_{1} = \ang{0}$ \\ \hline
$\va{F}_{2}$ & $6$ & $\theta_{2} = \ang{40}$ \\ \hline
$\va{F}_{3}$ & $3$ & $\theta_{3} = \ang{30}$ \\ \hline
$\va{F}_{4}$ & $5$ & $\theta_{4} = \ang{90}$ \\ \hline
\end{tabular}
\end{table}
\end{frame}
\begin{frame}
\frametitle{Calculando las componentes}
Una vez que tenemos la lista de vectores, iniciamos el cálculo de las componentes en las direcciones $x$ e $y$ de cada uno de ellos.
\\
\bigskip
\pause
Siendo recomendable también hacer una tabla con las componentes.
\end{frame}
\begin{frame}
\frametitle{Tabla con las componentes}
\begin{table}
\centering
\begin{tabular}{c | l | l | c}
Componente & Expresión & Sustitución & Valor \\ \hline
$F_{1x}$ & $\cos \theta_{1} \cdot F_{1}$ & $\cos \ang{0} \cdot \SI{8}{\newton}$ & $\SI{8}{\newton}$ \\ \hline
$F_{1y}$ & $\sin \theta_{1} \cdot F_{1}$ & $\sin \ang{0} \cdot \SI{8}{\newton}$ & $\SI{0}{\newton}$ \\ \hline
$F_{2x}$ & $\cos \theta_{2} \cdot F_{2}$ & $\cos \ang{40} \cdot \SI{6}{\newton}$ & $\SI{4.59}{\newton}$ \\ \hline
$F_{2y}$ & $\sin \theta_{2} \cdot F_{2}$ & $\sin \ang{40} \cdot \SI{6}{\newton}$ & $\SI{3.85}{\newton}$ \\ \hline
\end{tabular}
\end{table}
\end{frame}
\begin{frame}
\frametitle{Tabla con las componentes}
\begin{table}
\centering
\begin{tabular}{c | l | l | c}
Componente & Expresión & Sustitución & Valor \\ \hline
$F_{3x}$ & $-\cos \theta_{3} \cdot F_{3}$ & $-\cos \ang{30} \cdot \SI{3}{\newton}$ & $-\SI{2.59}{\newton}$ \\ \hline
$F_{3y}$ & $-\sin \theta_{3} \cdot F_{3}$ & $-\sin \ang{30} \cdot \SI{3}{\newton}$ & $-\SI{1.5}{\newton}$ \\ \hline
$F_{4x}$ & $\cos \theta_{4} \cdot F_{4}$ & $\cos \ang{90} \cdot \SI{5}{\newton}$ & $\SI{0}{\newton}$ \\ \hline
$F_{4y}$ & $\sin \theta_{4} \cdot F_{4}$ & $\sin \ang{90} \cdot \SI{5}{\newton}$ & $\SI{5}{\newton}$ \\ \hline
\end{tabular}
\end{table}
\end{frame}
\begin{frame}
\frametitle{Recuperando las componentes del vector resultante}
Ahora ya podemos calcular las componentes del vector resultante, \pause recordemos que:
\pause
\begin{eqnarray*}
\begin{aligned}
F_{Rx} &= \nsum_{i=1}^{4} F_{ix} = \\[0.5em] \pause
&= F_{1x} + F_{2x} + F_{3x} + F_{4x} = \\[0.5em] \pause
&= \SI{8}{\newton} {+} \SI{4.59}{\newton} {+} (-\SI{2.59}{\newton}) {+} \SI{0}{\newton} = \\[0.5em] \pause
&= \SI{9.69}{\newton}
\end{aligned}
\end{eqnarray*}
\end{frame}
\begin{frame}
\frametitle{Recuperando las componentes del vector resultante}
Para la componente $F_{Ry}$:
\pause
\begin{eqnarray*}
\begin{aligned}
F_{Ry} &= \nsum_{i=1}^{4} F_{iy} = \\[0.5em] \pause
&= F_{1y} + F_{2y} + F_{3y} + F_{4y} = \\[0.5em] \pause
&= \SI{0}{\newton} {+} \SI{3.85}{\newton} {+} (-\SI{1.5}{\newton}) {+} \SI{5}{\newton} = \\[0.5em] \pause
&= \SI{7.35}{\newton}
\end{aligned}
\end{eqnarray*}
\end{frame}
\begin{frame}
\frametitle{Magnitud del vector resultante}
Una vez obtenidas las componentes en $x$ e $y$, calculamos la magnitud del vector resultante:
\pause
\begin{eqnarray*}
\begin{aligned}
\abs{F_{R}} &= \sqrt{(F_{Rx})^{2} + (F_{Ry})^{2}} = \\[0.5em] \pause
&= \sqrt{(\SI{9.69}{\newton})^{2} + (\SI{7.35}{\newton})^{2}} = \\[0.5em] \pause
&= \sqrt{\SI{93.89}{\square\newton} + \SI{54.02}{\square\newton}} = \\[0.5em] \pause
&= \sqrt{\SI{147.91}{\square\newton}} = \\[0.5em] \pause
&= \SI{12.16}{\newton}
\end{aligned}
\end{eqnarray*}
\end{frame}
\begin{frame}
\frametitle{El ángulo del vector resultante}
Tenemos que $F_{Rx} > 0$ y $F_{Ry} > 0$, por lo que el vector resultante está en el cuadrante I, \pause con esos valores podemos calcular el valor del ángulo del vector resultante:
\pause
\begin{eqnarray*}
\begin{aligned}
\tan \theta_{R} &= \dfrac{F_{Ry}}{F_{Rx}} = \pause
 \dfrac{\SI{7.35}{\newton}}{\SI{9.69}{\newton}} = \pause  0.7585 \\[0.5em] \pause
\arctan(\tan \theta_{R}) &= \arctan(0.7585) \\[0.5em] \pause
\theta_{R} &= \ang{37.18}
\end{aligned}
\end{eqnarray*}
\end{frame}
\end{document}